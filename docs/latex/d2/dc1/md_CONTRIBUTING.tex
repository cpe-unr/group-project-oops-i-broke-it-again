Use feature branching to contribute to the project. This should keep conflicts to a minimal.

Our branch names will look something like\+:
\begin{DoxyItemize}
\item {\ttfamily master}\+: Branch reflecting the current working verions. Don\textquotesingle{}t commit directly to master. This will be our branch that pull requests are merged to. Code changes are made in branchs and merged into development using a pull request. Use the following format when naming branches\+:
\begin{DoxyItemize}
\item {\ttfamily feature/some-\/branch-\/feature-\/description}
\end{DoxyItemize}
\end{DoxyItemize}

\#\+General Flow
\begin{DoxyItemize}
\item When creating a new feature\+:
\end{DoxyItemize}
\begin{DoxyEnumerate}
\item Pull from master to get latest changes\+: {\ttfamily git pull origin master}.
\item Create a new branch and switch over to it\+: {\ttfamily git checkout -\/b feature/some-\/branch-\/feature-\/description}.
\item Do work on branch. You can view which branch you are on via\+: {\ttfamily git status}.
\item When you feel your work is complete, you can merge to master.
\begin{DoxyItemize}
\item Make sure your branch has the latest master changes\+: ``` git checkout origin master git pull origin master ``{\ttfamily }
\item {\ttfamily Switch back to your feature branch and merge in master.}git merge master{\ttfamily }
\item {\ttfamily Commit and push changes to github.}git push some-\/feature-\/branch`
\item Open a pull request via Git\+Hub and click merge.
\end{DoxyItemize}
\end{DoxyEnumerate}

This flow should prevent us from stepping on each others code and keeps all merge conflicts 